\documentclass[a4paper]{article}

\usepackage[sort]{natbib}
\usepackage{fancyhdr}


% \documentclass[a4paper]{article}

\usepackage[english]{babel}
\usepackage[utf8]{inputenc}
\usepackage{amsmath}
\usepackage{graphicx}
\usepackage[colorinlistoftodos]{todonotes}
\usepackage{hyperref}
\usepackage{listings}
% \usepackage[numbers]{natbib}

\usepackage{graphicx}
\usepackage{babel,blindtext}

\usepackage{algorithm}
\usepackage[noend]{algpseudocode}






% you may include other packages here (next line)
\usepackage{enumitem}



%----- you must not change this -----------------
\oddsidemargin 0.2cm
\topmargin -1.0cm
\textheight 24.0cm
\textwidth 15.25cm
% \parindent=0pt
\parskip 1ex
\renewcommand{\baselinestretch}{1.1}
\pagestyle{fancy}
%----------------------------------------------------



% enter your details here----------------------------------

\lhead{\normalsize \textrm{Build a Game-Playing Agent - Heuristic Analysis}}
\chead{}
\rhead{\normalsize August 8, 2017}
\lfoot{\normalsize \textrm{AIND - Udacity}}
\cfoot{}
\rfoot{Uirá Caiado}
\setlength{\fboxrule}{4pt}\setlength{\fboxsep}{2ex}
\renewcommand{\headrulewidth}{0.4pt}
\renewcommand{\footrulewidth}{0.4pt}


\begin{document}


%----------------your title below -----------------------------

\begin{center}

{\bf \large Choosing a Heuristic to an Isolation Game-Playing Agent \\ \small Uirá Caiado}
\end{center}


%---------------- start of document body------------------

As explained by \cite{Udacity2017}, in this project we had to develop an adversarial search agent to play the game "Isolation".  Isolation is a deterministic, two-player game of perfect information in which the players alternate turns moving a single piece from one cell to another on a board.  Whenever either player occupies a cell, that cell becomes blocked for the remainder of the game.  The first player with no remaining legal moves loses, and the opponent is declared the winner.


According to \cite{russelartificial}, games like this one have been engaging many AI researchers due its abstract nature. The state of a game is easy to represent, and agents are usually restricted to a small number of actions whose outcomes are defined by precise rules. However, those games are often too hard to solve by simple search. For instance, chess has a search tree with $35^{100}$ nodes.

To complete this project, I implemented two strategies: minimax algorithm and alpha-beta prunning. As explained by \cite{russelartificial}, the first one performs a complete depth-first explporatiom of the game tree and its time cost in real games is impractical.

explains that heuristic evaluation functions allow us to approximate the true utility of a state without doing a complete search.



























% ----------------end of document body---------------------

%---------------- start of references------------------

\bibliographystyle{plain}
% or try abbrvnat or unsrtnat
\bibliography{biblio.bib}

%---------------- end of references------------------


\end{document}
